\begin{enumerate}[leftmargin=2cm,labelsep=.5cm,label=\bf\arabic*.]

\item Problem 11
\begin{enumerate}
  \item The maximum number of digital inputs that can be powered by a single digital output.
  \item The amount that a signal exceeds the threshold for a LOW or HIGH. 
  \item The maximum current sunk by the IC when it is outputting HIGH.
  \item The output voltage of an IC's HIGH level.
  \item The time interval between a digital input changing state and a digital output switching from LOW to HIGH. This delay is measured from the points at which the input and output are halfway between logical states.
  \item An electronic circuit that delivers an electric current. 
\end{enumerate}

\item Problem 16 \\
\emph{See addendum 1}

\item Problem 27 \\
\emph{See addendum 2}

\item Problem 45 \\
Tri-state devices do not use tricimal arithmetic because the third "state" is essentially an open circuit. Whereas a logical '0' still outputs some voltage, the 'Z' or high-impedance state outputs nothing at all, removing the output from the circuit. The purpose of a tristate device is to isolate a particular output from a circuit, allowing for multiple outputs to be connected to the same line. \\
\emph{See addendum 3}

\item Problem 48 \\
Static-0: An output should be 0, but temporarily goes to a 1 as an input changes. \\
Static-1: An output should be 1, but temporarily goes to a 0 as an input changes.

\end{enumerate}
